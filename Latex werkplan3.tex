
\documentclass[a4paper]{article}
\usepackage[usenames,dvipsnames]{color}
\usepackage{graphicx}  
\usepackage[fleqn]{amsmath}
\sloppy
\begin{document}
\title{Werkplan}
\author{
		\and
		Nils De Vries\\ 
        Studentnummer 5730236  \\
        Universiteit Utrecht
        }

\date{Begeleider: \\ 
Richard van Dongen \\
\today}
\maketitle

\newpage

\section{Inleiding} 
\label{sec:inleiding} 

\section{Theoretische verkenning}
\section{Verkennende vragen}
\subsection{Onderzoeksdoel }
Aantonen dat de lineaire absorptieco\"{e}ffici\"{e}nt $\tau$ afhangt van $\lambda$  op de volgende manier:
$$
\tau \propto C*{\lambda}^3
$$
Met C een constante die onder meer afhangt van de dichtheid van het materiaal en het atoomnummer Z.  
\section{Opstelling en proefmeting}
\subsection{De dode tijd}
\subsubsection{Vooraf}
We hebben gekozen voor methode 1: Door gebruik te maken van twee verschillende stroomsterkten \textit{I\textsubscript{1}} en \textit{I\textsubscript{2}}, zodanig dat \textit{I\textsubscript{2}} = 2 \textit{I\textsubscript{1}}kunnen we met behulp van de vergelijking:
\begin{align}
s=r(1-s\tau)
\end{align}
bepalen wat de dode tijd is.
We willen twee stroomsterktes kiezen \textit{I\textsubscript{1}} en \textit{I\textsubscript{2}} zo dat bij \textit{I\textsubscript{1}} de invloed van de dode tijd erg klein  is en bij \textit{I\textsubscript{2}} de invloed groter. Uit (1) kunnen we halen dat als de invloed van de dodentijd klein is, ongeveer geldt: 
\begin{align}
s_1\approx r_1
\end{align} 

Omdat \textit{I} recht evenredig moet zijn met \textit{r} kunnen we vervolgens ook uitrekenen wat \textit{r\textsubscript{2}} is.\\

We weten van het apparaat dat de dode tijd is de orde van 1 * 10\textsuperscript{-4} s is. Dit komt er op neer dat het theoretische maximum aantal metingen 10000/s is. Dit wordt natuurlijk nooit gehaald en dus moeten we zorgen dat het aantal pulsen bij onze stroomsterkte \textit{I\textsubscript{1}} hier ver onder ligt. We nemen nu als (zelf gekozen) vuistregel dat onder de 10 \% de dode tijd verwaarloosbaar invloed heeft en boven de 10 \% we er rekening mee moeten houden. Deze grens ligt dus bij de 1000 pulsen/s. Er zitten natuurlijk heel wat haken en ogen aan deze benaderingsmethode (er is tenslotte niet echt een scherpe grens bij de 1000 pulsen), maar aangezien het een \textit{benadering} is gebruiken we hier hem toch.\\
 
Bij een stroomsterkte \textit{I\textsubscript{1}} en \textit{I\textsubscript{2}} van respectievelijk 0.2 A en 0.4 A en een voltage van 17 kV krijgen we bij \textit{I\textsubscript{1}} ongeveer een aantal pulsen/s van 700 en bij \textit{I\textsubscript{2}} ongeveer een aantal pulsen/s van 1300. \\

Nu moeten we natuurlijk nog de onzekerheid in ons gemeten aantal pulsen weten. Dit experiment laat zich beschrijven als een poissonverdeling, een discrete kansverdeling waarbij de getelde voorvallen geen invloed op elkaar uitoefenen. Bij een poissonverdeling wordt de onzekerheid in de beste schatter \textit{p} gegeven door de wordel van \textit{p} : $\sqrt{\textit{p}}$. \\

Als je op het apparaat de stroomsterkte instelt op 0, blijft er toch een kleine reststroom lopen die zorgt dat er r\"{o}ntgenstraling wordt uitgezonden en dus gedetecteerd. Om de effecten van de reststroom te verdisconteren doen we ook een meting van het aantal pulsen per seconde met \textit{I} ingesteld op 0. Het gemiddeld aantal gemeten pulsen bij \textit{I} = 0 trekken we vervolgens af van ons gemeten aantal pulsen bij \textit{I\textsubscript{1}} en \textit{I\textsubscript{2}}. \\

Vervolgens kunnen we gaan meten. We stellen de detector in onder een hoek van 0 graden, omdat de intensiteit dan het hoogst is en halen het target weg zodat we daar geen reflecties van krijgen. We meten over en tijdsinterval van 30 seconde zodat we een gemiddeld aantal pulsen per seconde krijgen bij die stroomsterkte.\\
\subsubsection{Metingen}
We hebben de volgende metingen gedaan:
\begin{itemize}
	\item \textbf{Reststroom}\\
    \textit{Gemiddeld aantal pulsen over 30 seconde}: 23.33\\
	\textit{Met onzekerheid}: 23 $\pm$ 4.8 pulsen/s
	\item \textbf{\textit{I} = 0.2 A}\\
	\textit{Gemiddeld aantal pulsen over 30 seconde}: 757.8\\
	\textit{Gemiddeld aantal pulsen over 30 seconde zonder reststroom}: 734.47\\
	\textit{Met onzekerheid, s\textsubscript{1}} : 734.5 $\pm$ 27.1 pulsen/s
	\item \textbf{\textit{I} = 0.4 A}\\
	\textit{Gemiddeld aantal pulsen over 30 seconde}: 1382\\
	\textit{Gemiddeld aantal pulsen over 30 seconde zonder reststroom}: 1359\\
	\textit{Met onzekerheid, s\textsubscript{2}} : 1359 $\pm$ 36.9 pulsen/s
\end{itemize}
\subsubsection{Verwerking metingen}
Uit (2) vinden we dat \textit{r\textsubscript{1}} ongeveer gelijk is aan \textit{s\textsubscript{1}} en dus aan 734.5 $\pm$ 27.1 pulsen/s. En omdat \textit{I} lineair is met \textit{r} geldt \textit{r\textsubscript{2}} = 2 \textit{r\textsubscript{1}} en dus weten we nu :\\
$$ 
\textit{r\textsubscript{2}} = 2 * 734.5 = 1469  \pm 44.2   \text{ pulsen/s}
$$
Nu hebben we dus zowel s\textsubscript{2} als r\textsubscript{2} en met (1) kunnen we nu $\tau$ vinden. We schrijven eerste de formule om:
 $$
 \tau = \frac{r_2 - s_2}{s_2*r_2}
 $$
 Als we hier onze gevonden getallen invullen vinden we:
 $$
 \tau = 5.510... *10^{-5} \text{ s}
 $$
 De onzekerheid en het aantal significante cijfers kunnen we vinden met propagatie van onzekerheden. Hierbij defini\"{e}ren we $\sigma_r$ en $\sigma_s$ als de onzekerheid in respectievelijk r\textsubscript{2} en s\textsubscript{2}.
 $$
 \sigma_\tau=\sqrt{\left(\frac{\partial{ \tau(r_2, s_2)}}{\partial{ r_2}}*\sigma_r\right)^2+\left(\frac{\partial{ \tau(r_2, s_2)}}{\partial{ s_2}}*\sigma_s\right)^2}
 $$
 Als we hier onze gevonden getallen in invulen vinden we: 
 $$
 \tau = 6* 10^{-5} \pm 3* 10^{-5} \text{ s}
 $$
 
 Deze gevonden dode tijd komt  goed overeen met de dode tijd van 100 $\mu$s die door de makers van het apparaat wordt opgegeven. 
 \subsection{Bragg-Reflectie}
 \subsubsection{Vooraf}
 Bragg-Reflectie is het principe dat wanneer er electromagnetische straling door een kristalrooster gaat de golflengtes op zo'n manier reflecteren dat ze onder bepaalde hoeken constructief intefereren en onder alle andere hoeken de electromagnetische straling destructief intefereert. Dit gebeurt volgens de formule:
 \begin{equation}
 n*\lambda=2d*sin(\theta)
 \end{equation}
 Hierna kan, wanneer de waarde van \textit{d} (de roosterafstand tussen twe kristalvlakken) van het materiaal bekend is de golflengte bij elke hoek berekend worden {\color{red}(Bedoel je hier de golflengte waarbij brachreflectie optreedt?)}, zolang hogere ordes niet dubbel worden geteld. Met deze formule kan vervolgens het energiespectrum van de r\"{o}ntgenstraling bepaald worden:  
 \begin{equation}
 E=h*c/\lambda
\end{equation}
 
 
 We meten bij een voltage waarbij we een redelijk hoog aantal counts verwachten. We gaan voor een voltage van 27.5 \textit{KV}, waar het aantal counts hoog genoeg is voor een goed meting. We meten bij een stroomsterkte van 0,50 \textit{mA}, we meten met een hoekstap van 0,1 graden, bij elke hoek meten we 5 seconden, zodat een plotselinge uitslag in de meting niet een verstoring in de meting geeft. We meten een groot bereik zodat we zoveel mogelijk orde van maxima kunnen opvangen, dus meten we van 0 tot 30 graden, natuurlijk laten we de sensor en het target samen bewegen, want als alleen de sensor zou bewegen zou de electromganetische straling geen hoek maken met het target en zou Bragg-reflectie niet optreden en als alleen het target zou bewegen zou er geen straling vallen in de sensor. Deze meting importeren we naar de computer via het programma X-Ray Apparatus van de Julius Programs. De onzekerheid in het aantal counts wordt gegeven door de wortel van de beste schatter van het aantal counts, omdat we bezig zijn met een statisch telexperiment en dus werken met een poissonverdeling.\\

\subsubsection{Metingen}
{\color{red}(Moet niet eerst het plaatje? Want je bespreekt toch al dingen uit het plaatje)}
 In het geïmporteerde bestand zijn uitgespreide curves en duidelijke pieken te zien. Natuurlijk zit er een piek bij een uitwijking van 0-0,5 graden;  hier gaat immers de electromagnetische straling rechtdoor en zit dus het 0de orde maximum zit. Tevens zitten er tussen de 6 en 7 graden twee pieken vrij dicht bij elkaar, waarvan de achterste piek de  hogere piek is. Hier zit een uitgespreide curve{\color{red}(Wat is een uitgespreide curve?)} omheen, het eerste orde maximum. Rond de 12 en 14 graden zitten eveneens twee pieken, waarvan de achterste piek weer de grootste is en er wederom een uitgespreide curve om de pieken heen zit.\\
 
 De pieken worden veroorzaakt door de karakteristieke golflengtes van Molybdeen en de uitgespreide curve komt door de golflengtes van de Brehmsstraling die beduidend lagere counts hebben dan de karakteristieke golflengtes. De breedte van de pieken wordt veroorzaakt doordat het per karakteristieke golflengte, \textit{k\textsubscript{$\alpha$}} en \textit{k\textsubscript{$\beta$}}, niet een enkele golflengte is maar een paar golflengtes die een 0,0001 nm van elkaar verschillen, wat betekent dat de machine ook uitslaat 0,1 a 0,2 graden naast de daadwerkelijke piek, de breedte van de pieken is dus in overeenstemming met de theorie.{\color{red}(Dit hele verhaal snap ik niet helemaal :), wat zijn bijvoorbeeld \textit{k\textsubscript{$\alpha$}} en \textit{k\textsubscript{$\beta$}} en waarom slaat de machine dan uit. Misschien ligt het aan mij dat ik het niet snap hoor, maar misschien kan het ietsje uitgebreider de uitleg )}\\
\subsubsection{Verwerking metingen}
 Hieronder een afbeelding van de meting met bijgevoegde onzekerheid in de meetpunten.
 \begin{center}
 \includegraphics[	width=12cm,height=10cm]{grafiek_rontgenstraling_bragg_reflectie.pdf} %%afbeelding wel in zelfde map zetten anders gaat ie zeuren
 \end{center}
 De pieken zitten zoals te zien op de hiervoor beshreven locaties, de pieken zijn te bepalen uit de meetwaardes en af te lezen uit de grafiek. \\ 
 
 Door de gevonden hoeken van de maxima in te vullen in vergelijking 3 met de karakteristieke golflengtes van onze molybdeen anode {\color{red}(Wat zijn dat?)}kunnen we dus de roosterafstand van ons NaCl-kristal bepalen. \\
 We gebruiken hiervoor een gemiddelde van de waardes van de eerste en tweede orde van zowel \textit{k\textsubscript{$\alpha$}} als \textit{k\textsubscript{$\beta$}}.
 {\color{red}(Hier geldt hetzelfde :) wat zijn \textit{k\textsubscript{$\alpha$}} en \textit{k\textsubscript{$\beta$}})}\\
 
 De waardes voor de \textit{k\textsubscript{$\alpha$}} en \textit{k\textsubscript{$\beta$}}\footnote{bron: http://physics.nist.gov/PhysRefData/XrayTrans/Html/search.html} zijn resp: 
 \begin{align*}
 \textit{k\textsubscript{$\alpha$}}=\text{0.0709 \textit{nm}}\\
 \textit{k\textsubscript{$\beta$}}=\text{0.0632 \textit{nm}}
 \end{align*}

 De eerste orde piek van \textit{k\textsubscript{$\beta$}} is bij een hoek van 6.3$^{\circ}$ en eerste orde piek van \textit{k\textsubscript{$\alpha$}} is bij een hoek van 7$^{\circ}$
 De tweede orde piek van \textit{k\textsubscript{$\beta$}} is bij een hoek van 12.7$^{\circ}$ en de tweede orde piek van \textit{k\textsubscript{$\alpha$}} is bij een hoek van 14.4$^{\circ}$. {\color{red}(Hoe kom je aan deze waarden? Zijn het meetwaarden en moet er dan geen onzekerheid bij?)}\\
 Wanneer we verglijking 3 omschrijven krijgen we:
 $$
 d=\dfrac{n*\lambda}{2*sin(\theta)}
 $$
Dit invullen in vergelijking 3 geeft ons bij elk van deze vier hoeken:

Voor de eerste ordes:\\
 \begin{align*}
d=\dfrac{1*0.0632}{2*sin(6.3)}=\text{0.288 \textit{nm}}\\
d=\dfrac{1*0.0709}{2*sin(7.0)}=\text{0.323 \textit{nm}}\\
\end{align*}
   Voor de tweede ordes:\\
\begin{align*}
d=\dfrac{2*0.0632}{2*sin(12.7)}=\text{0.287 \textit{nm}}\\
d=\dfrac{2*0.0709}{2*sin(14.4)}=\text{0.285 \textit{nm}}\\
 \end{align*}

 
%mogelijkheid 2,dit heeft mijn voorkeur
 Er is geen onzekerheid in de hoek bekend, dus de roosterafstand d wordt berekend door het gemiddelde te nemen van alle berekende roosterafstanden d, maar omdat de waarde van de roosterafstand bij de eerste orde van \textit{k\textsubscript{$\alpha$}} zo afwijkt van de andere drie waardes, en omdat de hoek niet de helft is van de tweede orde hoek zoals bij \textit{k\textsubscript{$\beta$}}, hebben we besloten dat resultaat te verwerpen als een meetfout. Het gemiddelde van de overige waardes is: 
 $$\dfrac{0.288+0.287+0.285}{3}=\text{0.287 \textit{nm}}$$
 
 \section{Experiment} %% Dit is de tweede sectie
\label{sec:exp} %% Weer een label

\subsection{Theorie}
\subsection{Opstelling}
\subsection{Meetmethode}
\subsection{Tijdsplanning}
\section{Testmeting}
\label{sec:test}
\section{Afronding}
\begin{thebibliography}{99}
	\bibitem{prachand} Practicumhandleiding bij de proef X.
	\bibitem{bron} R. Bron, \emph{Een betrouwbare bepaling}, Februari 2013.
\end{thebibliography}
\end{document}








